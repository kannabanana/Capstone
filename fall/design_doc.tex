\documentclass[letterpaper,10pt,titlepage,journal,compsoc,draftclsnofoot,onecolumn]{IEEEtran}
\newcommand\tab[1][1cm]{\hspace*{#1}}
\linespread{1}

\usepackage{graphicx}                                        
\usepackage{amssymb}                                         
\usepackage{amsmath}                                         
\usepackage{amsthm}                                          

\usepackage{alltt}                                           
\usepackage{float}
\usepackage{color}
\usepackage{url}
\usepackage{listings}

\usepackage{balance}
\usepackage[TABBOTCAP, tight]{subfigure}
\usepackage{enumitem}
\usepackage{pstricks, pst-node}

\usepackage{geometry}
\usepackage{titling}
\geometry{textheight=8.5in, textwidth=6in}


\newcommand{\cred}[1]{{\color{red}#1}}
\newcommand{\cblue}[1]{{\color{blue}#1}}

\usepackage{hyperref}
\usepackage{geometry}

\def\name{Garrett Amidon}


%% The following metadata will show up in the PDF properties
\hypersetup{
  colorlinks = true,
  urlcolor = black,
  pdfauthor = {\name},
  pdfkeywords = {cs461 ''Senior Capstone''},
  pdftitle = {CS 461 Senior Capstone: Problem Statement},
  pdfsubject = {CS461 Senior Capstone},
  pdfpagemode = UseNone
}

\title{Energy Effeciency Center Website \\
	\large Software Design Description}
\author{Garrett Amidon, SR Kanna, James O'Neal\\ Team \#36, Team WinniethePooh}

\begin{document}
\begin{titlingpage}
    \maketitle
	\centering{}
    \begin{abstract}
        
        OSU's Energy Efficiency Center help manufacturing and industrial companies increase their productivity and reduce their energy footprint by producing reports for energy and productivity recommendations. These reports, projects and funds are maintained in their website. The website has been developed and maintained by several programmers. As a result, the website has become disorganized, difficult to update and use. In order to remedy these issues we will design a secure, user friendly website with good code practices for the Energy Efficiency Center. Furthermore, the website is not accessible from mobile devices which decreases productivity while on the job site. With enough time, we would like to create a secure mobile app for the client which they are able to remotely access. This document will show the options available to fix these issues and which we think are best. 
        
    \end{abstract}
\end{titlingpage}

\newpage

\tableofcontents{}

\newpage


\section{Overview}

\subsection{Scope}

This document will discuss the development of the design elements. It will provide a detailed description of nine elements. These elements make up the core features of the system. It will include the connections between functionalities, specific technology, and viewpoints for each feature.

\subsection{Purpose}

This document specifies content and organization of  the software design attributes. It will include explanations, implementation tools, and multiple viewpoints for each design element. It will outline the important aspects of the solution. 

\subsection{Intended Audience}

This document is intended for the developers,  clients, and managers of the Energy Efficiency Center project. The developers will use this document to guide them through the selection, organization and presentation of the design information. Clients and managers can use the design document as a guide for the future development process.

\subsection{Conformance}

The design documentation conforms to clauses 4 and 5. Requirements are signified by shall. 

\section{Definitions}

For the purpose of this standard, the following terms and definitions apply.

\begin{description}
\item[Agile: ] a software engineering project method style
\item[Waterfall: ] a software engineering project method style
\item[User Stories: ] A tool in Agile to understand what prioritize the user's needs.
\item[Energy Efficiency Center: ] The clients
\item[Security:] defense of computers against intrusion and unauthorized use of resources
\item[Login:] an act of logging in to a computer, database, or system
\item[User Interface:] the means by which the user and a computer system interact
\item[Projects and Tasks:] projects and tasks must be added, edited, and viewed
\item[Viewing User's Work Hours:] employee work hours must be recorded and restricted 
\item[Time Clock: ] employees should be able to clock in and out while simultaneously assigning hours worked
\item[Employee Records:] Display and edit employee's information
\end{description}

\section{Conceptual Model for Software Design Descriptions}
\subsection{Software Design In Context}

This section creates conceptual models for the intended audience. The conceptual model will act as a guide for the developers and can be referenced by clients and managers for a high-level understanding.

\includegraphics[scale=0.75]{UML}
\newline Fig. 1: above is a diagram depicting the depencies and connections between design elements.
\subsection{Software Design Descriptions Within the Life Cycle}

Login credentials are sent to the database to be verified. Upon valid response the login page will redirect to the homepage. The homepage will provide navigation to all other pages. The projects and tasks page will query the database for relevant information. The projects and tasks page will include access to clock functionalities. Clock in and out interface with projects and tasks in order to view and update them. Employee information also needs to connect with projects and tasks in order to display relevant data. All three functionalities interact with the database to read and update data. 

\section{Design Description Information Content}
\subsection{Introduction}

The required design elements are:
\begin{enumerate}
\item{Security}
\item{Login}
\item{User Interfaces}
\item{Projects and Tasks}
\item{Viewing User's Work Hours}
\item{Code Readability}
\item{Database Management}
\item{Time Clock}
\item{Employee Records}
\end{enumerate}
They are described in detail in the remainder of the clause.

\subsection{Security}

Security should be designed alongside the software. Threats to consider and our criteria for evaluation include malicious SQL injections, forgetting to update software, XSS, error messages, server side validation, passwords, file uploads and SSL among others. All of the security scanning software being considered is open-source, so price is not an issue. Netsparker addresses sql injection and XSS, but fails to focus the other main security threats. Netsparker is noted for it's ease of use. OpenVas is the most extensive of the three security scanning options, since it can test over 25000 vulnerabilities. However, it's difficulty to use and time consumption is notorious. Fiddler focuses on vulnerabilities in HTTP traffic, which does not address any criteria except authenticity. OpenVas has the most rigorous security scanning model and meets the most criteria. However, if it's too difficult and time-consuming to use, then Netsparker should be supplemented to the scanning process.

\subsection{Login}

Login in websites should prevent against common attacks such as sql injections, session hijacking, network eavesdropping, cross site scripting, brute force attacks,  and converting time channel attacks. The best method to prevent against these attacks is a combination of secure login can be obtained via php and mysql. This kind of method prevents against sql injections, session hijacking, network eavesdropping, cross site scripting, brute force attacks,  and converting time channel attacks. Furthermore, since I am experienced with sql and php, there would be a minimal learning curve and time wasted. Even though it can be tedious to build and maintain, it can be combined with the canvas email structure so users do not have create a second email.

\subsection{User Interfaces}

Users expect aesthetics and ease of use when interacting with web applications. CSS tools like Bootstrap, Foundation and Boilerplate provide structure, navigation and aesthetic appeal on all web platforms. All three are responsive web tools and all three provide HTML and CSS frameworks for forms, buttons, Javascript extensions, typography. Bootstrap and Foundation's offerings are more extensive than Boilerplate's. All three are supported by multiple browsers, but Bootstrap and Foundation are the most extensive. However, Boilerplate is the most lightweight platform option. Bootstrap has a large open-source platform and is geared towards engineers to use and contribute to. Foundation is primarily used by large corporations. I am also extremely familiar with Bootstrap, so there would be a minimal learning curve or time wasted. Under the categories of structure, aesthetics, time and responsiveness, I recommend using Bootstrap. 

\subsection{Projects and Tasks}

\tab A gantt chart for visualizing projects and tasks can be displayed using Google charts. This API designed by google runs using embedded javascript. Google charts is one of the simplest gantt chart API's because all that is needed is the google chart libraries and a database to populate the chart. One of the most important parts of this API its integrated cross-platform and cross-browser compatability. It is fully customizable with many useful features such as grouping tasks and task dependencies. The chart is rendered using SVG which is widely used and open source. Finally, the API is very well documented providing example code and tutorials.\newline

\tab Projects and tasks are vital to the operation and organization of the EEC. In order to stay on track and be productive employees need to keep track of projects and tasks. Additionally, they need a clear view of project timelines and status of specific tasks. All this information needs to be easily viewable and readily available. The best technology to implement a solution is Google Charts. The API looks great and has highly customizable files. It is excellently documented and includes tutorials and examples. It includes drag and drop functionality giving it high usability. It is built on SVG creating a stable open source platform that will be maintained for the future. Overall this API is better than the others because it is both simple to implement and use.

\subsection{Viewing User's Work Hours}

Tabulator is a jQuery plugin which includes features such as pagination and filtering. The biggest advantage of this table is that it allows for direct table input editing. This would allow end users to update and submit data within the table allowing for quicker updates. The table accepts a wide range of data inputs including Ajax, JSON, and HTML. Unfortunately, this plugin does not include customization options for its appearance. However, it has a simple design and includes graphics for added clarity. \newline

\tab Displaying employee hours is important for the employees and their supervisor. They need clear visuals that allow them to verify correct inputs. Additionally they will not want a complex or time consuming process for managing and updating their hours. To accomplish this the best technology to use is Tabulator. With tabulator the table view of hours is consistent and easy to read. It supports organization and pagination features which will make finding specific rows easy. Interactive editing will allow users to quickly update and manage their hours. Tabulator's documentation, features, and ease of use put it ahead of the other options.

\subsection{Code Readability}

\tab An option for improving software quality is Codacy. This software is an automated code review platform. It has multiple metrics which can be turned on or off depending on the type of analyzation that needs to be done. Some of the possible metrics are accuracy, complexity, and coding style. It is compatible with multiple languages including PHP and JavaScript. It also can be integrated with GitHub making it easy to check metrics as the project progresses. In addition to its thousands of rules it also offers feedback on security and performance. Codacy is also highly secure making it viable for projects like this where data may be sensitive. \newline

\tab Software quality is important for the website because we will not be maintaining it in the future. Employees of the company will need to take what we have built and understand it well enough to make changes. While code standards and metrics will be used while writing it a software tool could greatly increase our successfulness in meeting this requirement. We will be implementing Codacy into our final solution. This software provides endless rules to check our final code with. It also will not reduce productivity because it can be integrated straight into github. The best part about Codacy is its professional quality and ability to catch errors from start to finish.

\subsection{Database Management}

\par Structured Query Language, SQL, is a standard computer language for database management. SQL is used to create queries which will insert, update and modify data. SQL is one of the most wide-used database management systems and has the added benefit of being supported across several different platforms. SQL has a lot of advantages and very few disadvantages. One advantage to SQL is that it is high speed and able to retrieve large amounts of data efficiently. This means that the EEC will be able to rely on this database to perform under high stress. The one drawback is that SQL is a commonly known language, so it is extremely susceptible to injection attacks. It also costs money to have an SQL server.\newline
\par For the database type, going with SQL is the best. Because of its wide-use, the EEC already has an SQL server implemented. Considering there is no big drawbacks to using SQL, it seems unnecessary to force a company to learn something new. Also with SQL you can access your tables through MySQL, so you do not need to know how to query to select results or view database entries. Plus, if you do need to create a query, by using MySQL it will create the query for you and give you the code to use in your webpage. Overall, SQL will provide the best service and overall best experience for the EEC. 

\subsection{Time Clock}

\par The site needs a system that allows employees to clock in and clock out for certain projects, as long as they don't go over the amount of time allotted to that project. With a database, when a project is posted, it will log the start date and the amount of available hours on the project. After doing so, we can create a system of clocking in to a database and then keeping track of how many hours before clocking out. When they clock in, they will be notified of how many hours are available on the project and will clock them out if they go over the posted amount. With this system it will insure that no one over works a certain project and that the company won't be spending extra time on projects. Also, if the manager decides to change the amount of time on a project, this can easily be done so in the database. \newline


\par Again, the best way to handle this system is through database handling. The named time clocks are great for keeping track of when someone clocks in and out, but not for when someone goes over hours on a certain project. With a database, we can store when a project is started and how much time it can be worked on. After doing so, when an employee goes to clock in, it can notify them the available hours left on that project and won't allow them to go past the posted amount. 

\subsection{Employee Records}

\par The site needs to be able to keep track of employees. The page needs to include a list of  all active employees and link to pages that contain specific employee information. This employee information contains pay rate, major, the date they started at the center as well as other information. From these pages, you must be able to view and edit employee information. A database is an organized collection of data. The data is stored and accessed through tables and queries. By having a database, the EEC will be able to make changes to employee records by using their own webpage. To do so, they will need to use web forms and “Get” and “Post” requests. These requests will be able to receive and send data to and from their database securely and as frequently with no added cost to the company. There is also many different varieties of databases available to better service the needs of the EEC. Although, databases that are on personal websites are more susceptible to attacks so employees that manage the web page will have to know about security. \newline

\par Employee records are private and need to be accessed on the daily, so having your own database ensures no one else has access to these records and will always be available. By using a web page and forms to submit changes to your database, you are in control of security and can be used at your disposal.

\section{Design Viewpoints}

\subsection{Introduction}

The following viewpoints and design elements will be discussed in the following format:

\begin{enumerate}
\item{Interaction (5.2)}
\begin{itemize}
\item{Time Clock}
\item{Login}
\item{Employee Records}
\item{Projects and Tasks}
\item{User Interfaces}
\item{Viewing User's Work Hours}
\end{itemize}
\item{Dependency (5.3)}
\begin{itemize}
\item{Security}
\item{Login}
\item{Database Management}
\end{itemize}
\item{Functional (5.4)}
\begin{itemize}
\item{Security}
\item{Login}
\item{User Interfaces}
\item{Projects and Tasks}
\item{Viewing User's Work Hours}
\item{Code Readability}
\item{Database Management}
\item{Time Clock}
\item{Employee Records}
\end{itemize}
\item{Information (5.5)}
\begin{itemize}
\item{User Interfaces}
\item{Database Management}
\end{itemize}
\item{Development (5.6)}
\begin{itemize}
\item{Code Readability}
\item{Security}
\item{Database Management}
\end{itemize}
\end{enumerate}



\subsection{Interactions Viewpoint}

\tab The interactions viewpoint will talk about the different interactions the user has with different areas of our project. These interactions include viewing or making actual interactions with the web page.

\subsubsection{Design Concerns}

\tab An easy and visually appealing experience for the user is a high priority for this project. The user should be able to interact with different areas of the web page and get a quick response from the web page. A concern here is knowing the cut off between feedback and usability and how the user will interact with different features. If a feature is cost effective and may take longer to display the information, we may need to substitute how it looks for response time.

\subsubsection{Design Elements}

\begin{enumerate}

\item{Time Clock}
\item{Login}
\item{Employee Records}
\item{Projects and Tasks}
\item{User Interfaces}
\item{Viewing User's Work Hours}

\end{enumerate}

\subsubsection{Interaction Attributes}

\tab The user must be able to log hours into projects they have worked on. Also, users should be notified if they have exceeded the allotted amount of time for that project. To do this, the user will have the option to enter their project hours by interacting with a web page. From there, the webpage will send a request to the database and the database will send a notification back. This notification will either tell the user that everything was submitted successfully, exceeded the allotted time, or if there was an error.\newline


\tab When landing on the Energy Efficiency Center's web page, you should always land on the Login page. The login page will ask the user for their username and password before letting them proceed any further. Upon hitting submit, the web page will use php to check for potential security risks and send the (safe) information securely to the database and check to make sure it is valid using SQL. If the login information is invalid, a notification will be sent back to the user saying so and will not let them move any further into the webpage. If the login information is valid, the database will send a notification back to the webpage and then the webpage will redirect the user to the homepage.\newline


\tab The employee records page should act as a “profile” page. This page will display the user's personal information such as their name, contact information, start date, position in company, and projects worked on. Upon landing on this page, the web page will use php to send an SQL query to the database to retrieve the current logged in user's information. The database will then send this information back to the web page and will lay it out for the user to interact with. At the top of the page the user will see their name and personal information and have the option to edit their contact information if needed. The user will also be able to view their most recent projects they have worked on, as well as have the option to view all their projects.  \newline


\tab The projects page should be the landing page that the user ends up on when trying to view all the available projects. From there the user will be able to click on any of the projects and be brought to a new page that displays the Gantt chart, using Google Charts, as well as a list of all the tasks for this project. When clicking on a task, it will show the description of the task, allotted hours for this task, and how many hours have been logged on this task. \newline


\tab On every web page, the user will have some sort of interaction with the user interface. Whether it be viewing or navigating, it should be an easy and visually appealing process. \newline


\tab The user will be able to view their work hours. These work hours should should users which tasks they have worked on and for how long. It should also show the user the total number of hours they have worked. \newline

\subsection{Dependency Viewpoint}

\tab The dependency viewpoint specifies the relationship of interconnection and access among entities. These relationships include shared information, order of execution or parameterization of interfaces.

\subsubsection{Design Concerns}

\tab
Security will be a main priority to keep the integrity of sensitive client data. To ensure the integrity and confidentiality, authenticity of users must also be verified. A secure login is the best way to do is. Therefore, login is dependent on security.\newline


\tab
The user interface determines how the employee records, clock in/out, and project and task management layout will be. For example, the current project management system is extremely difficult to navigate and update. By focusing on user interface, we can use calender flow-charts to make sure the clients can easily navigate the new system. To ensure the optimal utility, the database management will also dictate functionality and layout of the requested client features. In the current website, employees may update their information in one page, but the information in the database will not update in another section of the webpage. 

\subsubsection{Design Elements}

\begin{enumerate}
\item{Security}
\item{Login}
\item{User Interfaces}
\item{Database Management}
\item{Employee Records}
\item{Projects and Tasks}
\item{Time Clock}
\item{Code Readability}
\end{enumerate}

\subsubsection{Dependencies Attributes}

\tab Security keeps the integrity of sensitive client data. To ensure the integrity and confidentiality, authenticity of users must also be verified. A secure login is the best way to do is. Therefore, login is dependent on security.\newline


\tab User Interface and Database Management determines how the employee records, clock in/out, and project and task management layout will be.  By focusing on user interface, we can use calender flow-charts to make sure the clients can easily navigate the new system. In the current website, employees may update their information in one page, but the information in the database will not update in another section of the webpage. \newline

\subsection{Functional Viewpoint}

The functional viewpoint describes the system's functional elements, their responsibilities, interfaces, and primary interactions. As well as focuses on the system as a whole and it's ability to change, secured and runtime performance.

\subsubsection{Design Concerns}

To make everything functional, it is a concern that we will be able to create an experience for users that is both visually appealing and have reliable performance. Everything should function properly and the user's view should not look cluttered or unappealing. 

\subsubsection{Design Elements}

\begin{enumerate}
\item{Security}
\item{Login}
\item{User Interfaces}
\item{Projects and Tasks}
\item{Viewing User's Work Hours}
\item{Code Readability}
\item{Database Management}
\item{Time Clock}
\item{Employee Records}
\end{enumerate}

\subsubsection{Functional Attributes}

Making sure the system as whole is secure is crucial to the overall functionality of the system. If the system is not secure, it will susceptibil to attacks that could render the system useless. To defend against this, we will make sure all features are secure and only authorized users can access the system.  


Alongside security, having a login page be the initial landing page of the system will help make sure unauthorized users do not have access to the system. By making sure only authorized users can access the website, it makes sure the databases are more protected against attacks and private information cannot be reached. 


By making a user interface that is responsive and reliable, it ensures the system as a whole is functioning at its best. Also since the user interface is how all interactions between the system and the user are handled, it is crucial that it is always functioning at its peak. 


One of the features of the system is to be able to view and edit projects and tasks. To handle this, we will have a page that shows a list of all the projects and edit projects. To get this information, the page will be connected to a database with all the needed information. When the user clicks on a project, it will show all the tasks that are in the project and give the option to edit the project. 


Another feature is for users to be able to clock in hours to specific tasks they have worked on. To do this, when searching for projects and tasks, upon finding and clicking on a specific project, they will be able to enter hours worked on different tasks. These inputs are then sent to a database securely so they can be reached again at a later time. 


The user should also be able to view their work hours. To see their work hours, users will navigate to a page that gives a description and number of hours worked on all the tasks they have worked on, as well as total number of hours. To do this, when a user logs their hours on a certain task, it will be stored with their specific user id. So when retrieving the tasks, we will search for the specific user id and get the needed information. By doing it this way, it ensures that all hours will be accounted for and the user will always have access to their hours. 


To make sure the system as a whole can be easily changed, we will make sure the code is well commented and easy to follow so the code is easy to read. By doing this, the clients will be easily able to make changes at their discretion without having to know what all the code means. 


Since most of the functionality is done through the database, choosing a database that is reliable and usable is of highest importance. By going with an SQL database, it will make it easy to manage, retrieve and store information. The database will be primarily accessed by the user through other functionality on the web page. 


The final feature is the ability for user to look at their employee record. Upon clicking on the employee record option, it should show the user's name and contact information and the ability to change it, if needed. It will also show their most recent tasks they have clocked in on and the ability to check the rest. All this information will be retrieved from the SQL database.

\subsection{Information Viewpoint}

The information viewpoint describes the way that the architecture stores, manipulates, manages, and distributes information. The ultimate purpose of the system as a whole is to show the information in both a responsive and readable manner. 

\subsubsection{Design Concerns}

A design concern we have is to be able to store and manipulate user data and also to show the user their information. To store the data we will be using an SQL database so the user can access their data, edit it and store it again in an efficient manner. To allow the user to access this database, we will create a user interface that is both functional but visually appealing as well.

\subsubsection{Design Elements}

\begin{enumerate}

\item{User Interfaces}
\item{Database Management}
\end{enumerate}

\subsubsection{Information Attributes}

The database should be able to store the information in a way that makes the information retrievable and manipulatable. To best handle this, we will use a SQL database. An SQL database would make it so the information can be stored reliably and and be manipulated at the user's convenience. 

To make the information from the SQL database readable, we will create a user interface. This user interface will make it so the information is both received quickly after a request is made and distributed in a way that is visually appealing to the user. The user interface will also make it so the user can manipulate the information in the database easily and securely. 

\subsection{Development Viewpoint}

The development viewpoint describes the architecture that supports the software development process. Development views communicate the aspects of the architecture of interest to those stakeholders involved in building, testing, maintaining, and enhancing the system.

\subsubsection{Design Concerns}

A development concern that we have ie being able to create a system that is both secure, maintainable and responsive. To make sure the system is secure and maintainable, we will make sure the code is well commented and security is of highest priority. 

\subsubsection{Design Elements}

\begin{enumerate}

\item{Code Readability}
\item{Security}
\end{enumerate}


\subsubsection{Development Attributes}

To make sure the system is maintainable and secure, we will make sure the code readability and security are of highest priority. For code readability, we will make sure the code is easy to follow and well commented so the client will be able to make changes and updates seamlessly and efficiently. 

For the security portion, we will defend against all types of attacks to make sure the system can remain reliable. To help with this, we will use php to defend against possible database attacks and to check for unauthorized access to make sure information maintains safe.

\section{Signatures\newline\newline}

\noindent\begin{tabular}{ll}\newline
\makebox[2.5in]{\hrulefill} & \makebox[2.5in]{\hrulefill}\\
Anya Lehman & Date\\[8ex]% adds space between the two sets of signatures
\makebox[2.5in]{\hrulefill} & \makebox[2.5in]{\hrulefill}\\
Sr Kanna & Date\\[8ex]
\makebox[2.5in]{\hrulefill} & \makebox[2.5in]{\hrulefill}\\
Garrett Amidon & Date\\[8ex]
\makebox[2.5in]{\hrulefill} & \makebox[2.5in]{\hrulefill}\\
James O'Neal & Date\\
\end{tabular}


\end{document}